%++++++++++++++++++++++++++++++++++++++++
\documentclass[letterpaper,12pt]{article}
\usepackage{tabularx} % extra features for tabular environment
\usepackage[margin=1in,letterpaper]{geometry} % decreases margins
\usepackage{cite} % takes care of citations

%++++++++++++++++++++++++++++++++++++++++
\begin{document}
%Header-Make sure you update this information!!!!
\noindent
\large\textbf{Introduction to Service Design Engeneering Lab Report}\\

\hfill Lab Date: 22/11/2018 \\
Diaconu Marian \\
Molignoni Fabio \\
Gobber Rupert \\


%++++++++++++++++++++++++++++++++++++++++
\section*{Introduzione al laboratorio}
Il nostro laboratorio è stato incentrato su un'iniziale sezione di teoria, volta a fare mente locale sul funzionamento delle chiamate REST e sulle operazioni CRUD.
Successivamente è stato fatto un esercizio guidato per introdurre gli studenti a Ruby on Rails. Infine è stato assegnato un esercizio da iniziare in classe ed eventualmente finire a casa, offrendo il nostro supporto in caso di necessità.

%++++++++++++++++++++++++++++++++++++++++
\section*{Punti di forza}
La semplicità e la rapidità con la quale si può creare un servizio REST con Ruby on Rails è stato apprezzato da gran parte degli studenti. Le slides messe a disposizione e le referenze alla documentazione di Rails hanno permesso di riuscire a capire l'esercizio nonostante il framework fosse sconosciuto a molti.

%++++++++++++++++++++++++++++++++++++++++
\section*{Punti di debolezza}
Sicuramente  la scelta di utilizzare il framework Ruby on Rails ha disorientato gli studenti che non conoscevano il linguaggio. Tuttavia, è stata una buona occasione per vedere un nuovo tool molto adatto a questo contesto. 

%++++++++++++++++++++++++++++++++++++++++
\section*{Possibili miglioramenti}
La parte di REST poteva estendersi ancora per molto, essendo un argomento vasto e pieno di nozioni. Quindi in un eventuale futuro laboratorio sarebbe interessante pensare di dedicarci più tempo al fine di riuscire a fare una lezione con più nozioni ed esercizi.
 
%++++++++++++++++++++++++++++++++++++++++
\end{document}
